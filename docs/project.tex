\documentclass[10pt]{article}

% Lines beginning with the percent sign are comments
% This file has been commented to help you understand more about LaTeX

% DO NOT EDIT THE LINES BETWEEN THE TWO LONG HORIZONTAL LINES

%---------------------------------------------------------------------------------------------------------

% Packages add extra functionality.
\usepackage{times,graphicx,epstopdf,fancyhdr,amsfonts,amsthm,amsmath,algorithm,algorithmic,xspace,hyperref}
\usepackage[left=1in,top=1in,right=1in,bottom=1in]{geometry}
\usepackage{sect sty}	%For centering section headings
\usepackage{enumerate}	%Allows more labeling options for enumerate environments 
\usepackage{epsfig}
\usepackage[space]{grffile}
\usepackage{booktabs}
\usepackage{forest}
\usepackage{graphicx}


% This will set LaTeX to look for figures in the same directory as the .tex file
\graphicspath{.} % The dot means current directory.

\pagestyle{fancy}

\lhead{Final Project}
\rhead{\today}
\lfoot{CSCI 334: Principles of Programming Languages}
\cfoot{\thepage}
\rfoot{Fall 2023}

% Some commands for changing header and footer format
\renewcommand{\headrulewidth}{0.4pt}
\renewcommand{\headwidth}{\textwidth}
\renewcommand{\footrulewidth}{0.4pt}

% These let you use common environments
\newtheorem{claim}{Claim}
\newtheorem{definition}{Definition}
\newtheorem{theorem}{Theorem}
\newtheorem{lemma}{Lemma}
\newtheorem{observation}{Observation}
\newtheorem{question}{Question}

\setlength{\parindent}{0cm}


%---------------------------------------------------------------------------------------------------------

% DON'T CHANGE ANYTHING ABOVE HERE

% Edit below as instructed

\begin{document}
  
\section*{Project Proposal}

Samuel Xiang

\subsection{Introduction}

    \quad The language I would like to design would solve the "problem" of designing turn-based RPG games on a computer.
This language would be able to generate a playable game with an interactable world and combat between characters,
while giving the programmer plenty of flexibility to add to the existing structure as they please. In essence, it 
would provide the programmer with an outlet that allows them to express complex creative ideas in simple ways.
    \\ \\ \quad There are a vast selection of unique turn-based RPG's in the real world, but the genre has rules and restrictions 
of form that are almost impossible to break. As such, expressing the basic tenets of any turn-based RPG can be boiled
down to several patterns, which mean that a formal grammar is perfectly capable of encapsulating these tenets. However,
building these core ideas over and over again is redundant and time consuming, so a programming language should be a 
fun way to create these games with minimal labor. 

\subsection{Design Principles}

    \quad This language should express lengthy ideas and systems of turn-based RPG games in simple English. From the technical
perspective, this program should generate a playable game to the users specifications within the scope of the language.
By playable, I mean a game that an outsider user interact with and complete without knowledge of the inner workings - like
an actual video game. Because these ideas are expressable in simple English, I would like to create a readable program without 
lots of weird characters that almost anyone could access and easily intuit. Because the creation of a game involves the creation 
of a fictional world, the program should almost read as a story or encyclopedia of this world. 

\subsection{Examples}




\subsection{Language Concepts}

\quad    The language as I envision it right now doesn't actually have many moving parts - the set of primitives
I want combine nicely, and the most of the functionality/computation is hidden from the programmer. As such,
the programmer simply needs ideas to fill out the game. Creating the map, executing combat, etc. are all things 
that the programmer doesn't consider - rather, they create the guidelines that the functionality of the language
operates within. The most important things the programmer has to understand are how rooms are connected and how
stats interact in and outside of combat. 

\subsection{Syntax/Grammar}

\begin{verbatim}
<paragraph> ::= <sentence>
              | <sentence><paragraph>
<sentence>  ::= <character>
<character> ::= There is a <type> named <name> with stats: <stats> and abilities: 
    <abilities>.
<type>      ::= playable
              | nonplayable
<name>      ::= <string>
<string>    ::= *a string in F#*
<stats>     ::= hp = <int>, mp = <int>, atk = <int>, def = <int>, matk = <int>, 
    mdef = <int>, spd = <int> 
<int>       ::= *an int in F#*
<abilities> ::= <ability>
              | <ability>, <abilities>
<ability>   ::= <name> has effect <effect>
<effect>    ::= <string>
\end{verbatim}

\newpage

\subsection{Semantics}

\begin{table}[]
    \resizebox{\columnwidth}{!}{%
    \begin{tabular}{@{}|l|l|l|l|l|@{}}
    \toprule
    Syntax & Abstract Syntax & Type & Prec/Assoc & Meaning \\ \midrule
    \textless{}Sentence\textgreater (one or more) & Paragraph of Sentence list & Sentence list & n/a & Paragraph is a list of one or more sentences. \\ \midrule
    There is \textless{}type\textgreater character named \textless{}name\textgreater with stats: \textless{}stats\textgreater and abilities: \textless{}abilities\textgreater{}. & Character of record & Sentence & n/a & \begin{tabular}[c]{@{}l@{}}Character of record takes in four types and combines them in a record.\\ The primitive types are Type, Name, Stats, Abilities. If these types are\\ not present, an error will be thrown.\end{tabular} \\ \midrule
    \textless{}type\textgreater{} & Type of string & string & n/a & \textless{}type\textgreater is a primitive, represented by a standard F\# string. \\ \midrule
    \textless{}name\textgreater{} & Name of string & string & n/a & \textless{}name\textgreater is a primitive, represented by a standard F\# string. \\ \midrule
    hp = \textless{}int\textgreater{}, mp = \textless{}int\textgreater{}, atk = \textless{}int\textgreater{}, def = \textless{}int\textgreater{}, matk = \textless{}int\textgreater{}, mdef = \textless{}int\textgreater{}, spd = \textless{}int\textgreater{} & Stats of record & record & n/a & \textless{}stats\textgreater is a primitive, represented by a record of ints. \\ \midrule
    \textless{}Ability\textgreater (z eroor more) & Abilities of Ability list & Ability list & n/a & Abilities is a list of zero or more sentences. \\ \midrule
    ability \textless{}name\textgreater has effect \textless{}effect\textgreater{} & Ability of record & record & n/a & \begin{tabular}[c]{@{}l@{}}Ability is a record of Name and Effect. If these types are not present,\\ an error will be thrown.\end{tabular} \\ \midrule
    \textless{}effect\textgreater{} & Effect of string & string & n/a & \textless{}effect\textgreater is a primitive, represented by a standard F\# string. \\ \bottomrule
    \end{tabular}%
    }
\end{table}


% DO NOT DELETE ANYTHING BELOW THIS LINE
\end{document}
